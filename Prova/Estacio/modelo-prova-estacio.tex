%\documentclass[pdftex,a4paper]{article}
\documentclass[a4paper]{article}
%%classes: article, report, book, proc, amsproc

%%%%%%%%%%%%%%%%%%%%%%%%
%% Misc
% para acertar os acentos
\usepackage[brazilian]{babel} 
%\usepackage[portuguese]{babel} 
% \usepackage[english]{babel}
% \usepackage[T1]{fontenc}
% \usepackage[latin1]{inputenc}
\usepackage[utf8]{inputenc}
\usepackage{indentfirst}
\usepackage{fullpage}
% \usepackage{graphicx} %See PDF section
\usepackage{multicol}
\setlength{\columnseprule}{0.5pt}
\setlength{\columnsep}{20pt}
%%%%%%%%%%%%%%%%%%%%%%%%
%%%%%%%%%%%%%%%%%%%%%%%%
%% PDF support

\usepackage[pdftex]{color,graphicx}
% %% Hyper-refs
\usepackage[pdftex]{hyperref} % for printing
% \usepackage[pdftex,bookmarks,colorlinks]{hyperref} % for screen

%% \newif\ifPDF
%% \ifx\pdfoutput\undefined\PDFfalse
%% \else\ifnum\pdfoutput > 0\PDFtrue
%%      \else\PDFfalse
%%      \fi
%% \fi

%% \ifPDF
%%   \usepackage[T1]{fontenc}
%%   \usepackage{aeguill}
%%   \usepackage[pdftex]{graphicx,color}
%%   \usepackage[pdftex]{hyperref}
%% \else
%%   \usepackage[T1]{fontenc}
%%   \usepackage[dvips]{graphicx}
%%   \usepackage[dvips]{hyperref}
%% \fi

%%%%%%%%%%%%%%%%%%%%%%%%


%%%%%%%%%%%%%%%%%%%%%%%%
%% Math
\usepackage{amsmath,amsfonts,amssymb}
% para usar R de Real do jeito que o povo gosta
\usepackage{amsfonts} % \mathbb
% para usar as letras frescas como L de Espaco das Transf Lineares
% \usepackage{mathrsfs} % \mathscr

% Oferecer seno e tangente em pt, com os comandos usuais.
\providecommand{\sin}{} \renewcommand{\sin}{\hspace{2pt}\mathrm{sen}}
\providecommand{\tan}{} \renewcommand{\tan}{\hspace{2pt}\mathrm{tg}}

% dt of integrals = \ud t
\newcommand{\ud}{\mathrm{\ d}}
%%%%%%%%%%%%%%%%%%%%%%%%



\begin{document}

%%%%%%%%%%%%%%%%%%%%%%%%
%% Título e cabeçalho
\noindent\parbox[c]{.15\textwidth}{\includegraphics[width=.15\textwidth]{logo}}\hfill
\parbox[c]{.825\textwidth}{\raggedright%
  \sffamily {\LARGE

PRONATEC Matemática Aplicada: AV x

\par\bigskip}
{Estácio -- PRONATEC\par} 
%{\url{http://sites.google.com/site/proffelipefigueiredo}}

{Aluno(a): \underline{\hspace{8.5cm}} Nota: \underline{\hspace{2cm}}\par}%RA: \underline{\hspace{2.3cm}}\par}
{Curso: \underline{\hspace{8.95cm}} Data: \underline{\hspace{2cm}}\par}
{Turma: \underline{\hspace{2cm}} Sala: \underline{\hspace{2cm}}
  Matrícula: \underline{\hspace{4cm}} \par}
{Prof: Felipe Figueiredo\par}
%Série: \underline{\hspace{2cm}} Turno: \underline{\hspace{2cm}}\par

\vspace{1cm}
}
%%%%%%%%%%%%%%%%%%%%%%%%

Leia as questões com atenção. Faça resoluções claras e com letra
legível. 

Boa prova!

\underline{\hspace{15.5cm}}

%%%%%%%%%%%%%%%%%%%%%%%%

\begin{multicols}{2}
% {\bf Cartão resposta}

% \begin{tabular}[!b]{|c|c|c|c|c|}
%   \hline
%   1 & A & B & C & D\\
%   \hline
%   2 & A & B & C & D\\
%   \hline
%   3 & A & B & C & D\\
%   \hline
%   4& A & B & C & D\\
%   \hline
% \end{tabular}
% \begin{tabular}[!b]{|c|c|c|c|c|}
%   \hline
%   5 & A & B & C & D\\
%   \hline
%   6 & A & B & C & D\\
%   \hline
%   7& A & B & C & D\\
%   \hline
%   8 & A & B & C & D\\
%   \hline
% \end{tabular}

\begin{enumerate}
\item

\vspace{7cm}
\item 

\vspace{7cm}
\item 

\vspace{8.5cm}
\item 
\vspace{6cm}

{\ }
\columnbreak

\item 

\vspace{5cm}

\item

\vspace{6cm}
\item 

\vspace{7cm}

\vfill
\end{enumerate}
\end{multicols}

% \bigskip

% \underline{\hspace{15.5cm}}

% \bigskip
% {\bf Espaço extra disponível para rascunho}
% %\newpage

\end{document}
