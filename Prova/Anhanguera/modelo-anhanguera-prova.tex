\everymath{\displaystyle}
%\documentclass[pdftex,a4paper]{article}
\documentclass[a4paper]{article}
%%classes: article, report, book, proc, amsproc

%%%%%%%%%%%%%%%%%%%%%%%%
%% Misc
% para acertar os acentos
\usepackage[brazilian]{babel} 
%\usepackage[portuguese]{babel} 
% \usepackage[english]{babel}
% \usepackage[T1]{fontenc}
% \usepackage[latin1]{inputenc}
\usepackage[utf8]{inputenc}
\usepackage{indentfirst}
\usepackage{fullpage}
% \usepackage{graphicx} %See PDF section
\usepackage{multicol}
\setlength{\columnseprule}{0.5pt}
\setlength{\columnsep}{20pt}
%%%%%%%%%%%%%%%%%%%%%%%%
%%%%%%%%%%%%%%%%%%%%%%%%
%% PDF support

\usepackage[pdftex]{color,graphicx}
% %% Hyper-refs
\usepackage[pdftex]{hyperref} % for printing
% \usepackage[pdftex,bookmarks,colorlinks]{hyperref} % for screen

%% \newif\ifPDF
%% \ifx\pdfoutput\undefined\PDFfalse
%% \else\ifnum\pdfoutput > 0\PDFtrue
%%      \else\PDFfalse
%%      \fi
%% \fi

%% \ifPDF
%%   \usepackage[T1]{fontenc}
%%   \usepackage{aeguill}
%%   \usepackage[pdftex]{graphicx,color}
%%   \usepackage[pdftex]{hyperref}
%% \else
%%   \usepackage[T1]{fontenc}
%%   \usepackage[dvips]{graphicx}
%%   \usepackage[dvips]{hyperref}
%% \fi

%%%%%%%%%%%%%%%%%%%%%%%%


%%%%%%%%%%%%%%%%%%%%%%%%
%% Math
\usepackage{amsmath,amsfonts,amssymb}
% para usar R de Real do jeito que o povo gosta
\usepackage{amsfonts} % \mathbb
% para usar as letras frescas como L de Espaco das Transf Lineares
% \usepackage{mathrsfs} % \mathscr

% Oferecer seno e tangente em pt, com os comandos usuais.
\providecommand{\sin}{} \renewcommand{\sin}{\hspace{2pt}\mathrm{sen}}
\providecommand{\tan}{} \renewcommand{\tan}{\hspace{2pt}\mathrm{tg}}

% dt of integrals = \ud t
\newcommand{\ud}{\mathrm{\ d}}
%%%%%%%%%%%%%%%%%%%%%%%%

\begin{document}

%%%%%%%%%%%%%%%%%%%%%%%%
%% Título e cabeçalho
\noindent\parbox[c]{.15\textwidth}{\includegraphics[width=.15\textwidth]{logo}}\hfill
\parbox[c]{.825\textwidth}{\raggedright%
  \sffamily {\LARGE

Centro Universitário Anhanguera de Niterói -- UNIAN

\par\bigskip}
{Aluno(a): \underline{\hspace{8.5cm}} RA: \underline{\hspace{2.3cm}}\par}
{Curso: \underline{\hspace{8.95cm}} Data: \underline{\hspace{2cm}}\par}
{Turma: \underline{\hspace{2cm}} Série: \underline{\hspace{2cm}}
  Turno: \underline{\hspace{2.5cm}} Prova A\par}
{Prof: \underline{\hspace{9.15cm}} Nota: \underline{\hspace{2cm}}\par}

\vspace{10pt}
}
%%%%%%%%%%%%%%%%%%%%%%%%

{\large Prova x de DISCIPLINA}

\vspace{10pt}


%%%%%%%%%%%%%%%%%%%%%%%%
Leia atentamente as questões. Não serão consideradas respostas fora
dos espaços demarcados para as questões. Questões discursivas só serão
consideradas com resoluções completas. Questões objetivas só serão
consideradas no cartão resposta, se houver. Caso seja providenciada
uma folha de rascunho, entregá-la junto com a prova.

{\bf Não desgrampear a prova.} Não utilizar corretivo tipo {\em
  Liquid-Paper}.

Somente será permitido material de uso pessoal como lápis, borracha e
caneta sobre a mesa após o início da avaliação. Não será permitido o
uso de calculadoras, nem de dispositivos eletrônicos, como celular,
tablet, etc. A posse ou uso desses aparelhos durante a prova implicará
na anulação imediata da mesma. O mesmo vale para bolsas, estojos, etc.

% \vspace{0.1cm}
\hrulefill
% \vspace{0.3cm}

% (Enunciado geral, se houver)
% \vspace{0.5cm}

% %% linhas de preenchimento em página
% \underline{\hspace{15cm}}
% %% linhas de preenchimento em coluna
% \underline{\hspace{7cm}}

\begin{multicols}{2}
{\bf Cartão resposta} (preencher a opção)

\begin{tabular}[!b]{|c|c|c|c|c|}
  \hline
  1 & A & B & C & D\\
  \hline
  2 & A & B & C & D\\
  \hline
  3 & A & B & C & D\\
  \hline
  4& A & B & C & D\\
  \hline
\end{tabular}
\begin{tabular}[!b]{|c|c|c|c|c|}
  \hline
  5 & A & B & C & D\\
  \hline
  6 & A & B & C & D\\
  \hline
  7& A & B & C & D\\
  \hline
  8 & A & B & C & D\\
  \hline
\end{tabular}

\begin{enumerate}
%% Questões objetivas (espaço 0.1cm)
\item 
  \begin{enumerate}
  \item
  \item
  \item
  \item 
  \end{enumerate}

\vspace{0.1cm}

%% Questões discursivas (espaços grandes)
\item (1pt) 

\vspace{2cm}
\item (1pt) 

\vspace{3cm}
\item (2pt) 

\vspace{4cm}
\item (1pt) 

\vspace{3cm}
% {\ }
% \columnbreak

\item (1pt) 

\vspace{5cm}
\item (2pt) 

%\vspace{8cm}

\vfill
\end{enumerate}
\end{multicols}

\newpage
{\hfill Prova A\par}

\vfill
\begin{center}Boa prova!\end{center}
\end{document}
